\subsection{Purpose}

The purpose of this Requirements Analysis and Specification Document is to present the CodeKataBattle platform. CodeKataBattle (CKB) is a platform where students can improve their software development skills, by solving programming exercises, in this context called as katas.
Each kata is proposed by a single instructor throught the platform, and can be addressed by several groups of students in a programming language of choice.
katas are composed of the following elements:
\begin{itemize}
    \item A brief textual description of the problem
    \item A set of test cases that will the base to evaluate the proposed solution
\end{itemize}


The platform allows Educators to define deadlines on subscription and submission as long as other constraints such as the maximum and minimum number of students per group and other mechanisms for scoring. 

The problems are addressed in battles, organized in tournaments, where the groups of students can participate. 

The partecipation in a turnament can improve a student's rank on the platform and grant them badges, based on the performance measured.

The thresholds for badges, their release and the creation of other variables are features available only to educators, but their visualization is open to every other user.

The platform relies on GitHub actions for the creation of repositories containing code katas and the automatic testing of the proposed solutions. The students are required to fork and setup an automatic workflow through GitHubActions

\subsubsection{Goals}
Here we identify the goals of the system:

[G1] Students are able to practice on katas through participating to tournaments in groups

[G2] Students obtain a an automatic evaluation of their proposed solutions

[G3] Instructors are able to create, manage and award tournaments

[G4] Instructors are able to provide a manual evaluation of the proposed solutions


\subsection{Scope}

\subsubsection{World Phenomena}

\subsubsection{Shared Phenomena}

\subsection{Definitions, Acronyms, Abbreviations}

\subsection{Revision history}

\subsection{Reference Documents}

\subsection{Document Structure}
