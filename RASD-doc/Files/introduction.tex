This document has been prepared to help you approaching Latex as a formatting tool for your Travlendar+ deliverables. This document suggests you a possible style and format for your deliverables and contains information about basic formatting commands in Latex. A good guide to Latex is available here \href{https://tobi.oetiker.ch/lshort/lshort.pdf}{https://tobi.oetiker.ch/lshort/lshort.pdf}, but you can find many other good references on the web. 

Writing in Latex means writing textual files having a \texttt{.tex} extension and exploiting the Latex markup commands for formatting purposes. Your files then need to be compiled using the Latex compiler. Similarly to programming languages, you can find many editors that help you writing and compiling your latex code. Here \href{https://beebom.com/best-latex-editors/}{https://beebom.com/best-latex-editors/} you have a short oviewview of some of them. Feel free to choose the one you like.  

Include a subsection for each of the following items\footnote{By the way, what follows is the structure of an itemized list in Latex.}:
\begin{itemize}
\item
Purpose: here we include the goals of the project
\item
Scope: here we include an analysis of the world and of the shared phenomena
\item
Definitions, Acronyms, Abbreviations
\item
Revision history
\item
Reference Documents 
\item
Document Structure
\end{itemize}
Below you see how to define the header for a subsection.
\subsection{Purpose}
The purpose of this Requirements Analysis and Specification Document is to present the CodeKataBattle platform. CodeKataBattle (CKB) is a platform where students can improve their software development skills, by solving programming exercises, in this context called as katas.
Each kata is proposed by a single instructor throught the platform, and can be addressed by several groups of students in a programming language of choice.
katas are composed of the following elements:
\begin{itemize}
    \item A brief textual description of the problem
    \item A set of test cases that will the base to evaluate the proposed solution
\end{itemize}

The platform allows Educators to define deadlines on subscription and submission as long as other constraints such as the maximum and minimum number of students per group and other mechanisms for scoring.The problems are addressed in battles, organized in tournaments, where the groups of students can participate. 
The partecipation in a turnament can improve a student's rank on the platform and grant them badges, based on the performance measured.
The thresholds for badges, their release and the creation of other variables are features available only to educators, but their visualization is open to every other user.

The platform relies on GitHub actions for the creation of repositories containing code katas and the automatic testing of the proposed solutions.

\subsection{Scope}
... Here you see a subsubsection

\subsubsection{World Phenomena}

\subsubsection{Shared Phenomena}

\subsection{Definitions, Acronyms, Abbreviations}

\subsection{Revision history}

\subsection{Reference Documents}

\subsection{Document Structure}
